\vsssub
\subsubsection{~$S_{mud}$: Dissipation by viscous mud (D\&L)} \label{sec:BT8}
\vsssub

\opthead{BT8}{NRL/SWAN}{M. Orzech and E. Rogers}

\noindent
Two formulations for wave damping by viscous fluid mud have been implemented
in \ws\, based on earlier implementations in a SWAN code at NRL.  As with wave
damping by ice (Sect. \ref{sec:ICE1}), both rely on the concept of complex
wave number (Eq. (\ref{eq:waveno})). Both treat the mud layer as a viscous
fluid, and both assume that the mud depth is comparable to its Stokes'
boundary layer thickness. The first formulation (\cite{art:DL78}; hereafter
D\&L) is a numerical solution.  The second formulation (\cite{art:Ng00};
hereafter Ng) is an analytical, asymptotic solution, so calculations tend to
be much faster than with D\&L. For the range of mud characteristics used by
\cite{art:RH09}, which are based on field measurements (and estimates),
the methods produce very similar results.

In each case, the mud-induced dissipation is added to contributions from other
source/sink terms in Eq.~(\ref{eq:bal_plane}).

\begin{equation}\label{eq:dmud1}
  {S_{mud}} = 2  k_i {C_{g,mud}},
\end{equation}

\noindent
where $k_i=imag({k_{mud}})$ and ${C_{g,mud}}$ is the mud-modified wave group
velocity.

The above follows from exponential decay of a single wave train with initial
amplitude $a_0$:
\begin{equation}\label{eq:dmud2}
 a=a_0e^{-k_ix}.
\end{equation}

\noindent
Both methods operate by solving for a modified dispersion relation, where the
wavenumber being solved for, ${k_{mud}}$, is a complex number. The D\&L method
uses an iterative procedure for this dispersion relation. For details, see
Section 2 and Appendix B of \cite{art:DL78}. Descriptions specific to {\code
BT9} (Ng) are given in the following section.

To activate viscous mud effects with the (D\&L) routines, the user specifies
{\code BT8} in the switch parameter file.

In the case where any of the new ice and mud source functions are activated
with the switches {\code IC1}, {\code IC2}, {\code IC3}, {\code BT8}, or
{\code BT9}, {\file ww3\_shel} will anticipate instructions for 8 new fields
(5 for ice, then 3 for mud). These are given prior to the ``water levels''
information. The new fields can also be specified as homogeneous field using
{\file ww3\_shel.inp}. The mud parameters are mud density (kg/m$^3$), mud
thickness (m), and mud viscosity (m$^2$/s), in that order.

The user is referred to the regression tests {\file ww3\_tbt1.1} {\file
ww3\_tbt2.1} for examples of how to use the new mud source functions.

\textrm{\textit{\underline{Limitations of the code}}}: In the case of {\file ww3\_multi}, 
the interface for the necessary mud and ice forcing fields has only been implemented when using 
the namelist type of input file.  In the case of mud, though the ${k_r}$ is calculated, its effect is
not passed back to the main program. The only effect is via ${k_i}$
(dissipation). Full implementation of ${k_r}$, already possible with IC3, and
will be available in a future version of the model.

\textrm{\textit{\underline{Limitations of the physics}}}: 1) Both models
({\code BT8}, {\code BT9}) neglect elasticity in the mud layer. 2)
Non-Newtonian response of the mud (e.g. as a thixotropic fluid) is not
available. 3) Mud thickness should be interpreted not as the total mud
thickness, but rather as the thickness of the fluidized mud layer. This value
is notoriously difficult to determine in practice
(\cite{art:RH09}). Fortunately, since \ws\ supports nonstationary and
non-uniform input for the mud parameters, it is possible to address items (2)
and (3) via coupling with a numerical model of the mud layer: no additional
changes to the \ws\ code are required for this.
